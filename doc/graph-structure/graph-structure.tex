\documentclass{article}

\title{Impact of Network Topology on Cooperation in Public Goods Games}
\date{2018-11-11}
\author{John M. Maloney}

\usepackage{booktabs}
\usepackage{multirow}
\usepackage{amsmath}

\begin{document}

  \maketitle

  \section{Introduction}
  \paragraph{}Both evolutionary processes as well as economic marketplaces are driven by competition between individuals. It would appear that only selfish behavior would be rewarded.  However, cooperation and altruistic behavior is witnessed throughout the natural world and in human societies.  Explaining the existence of these cooperative and altruistic behaviors is a challenge for both evolutionary biology and economics.  If the cooperators are closely related, altruistic cooperation can be seen as benefiting reproduction of the shared gene set.  However, cooperation also occurs among unrelated individuals and this reciprocity-based cooperation is harder to explain.  
  \paragraph{}Lacking the ability to manipulate natural and economic systems in order to conduct experiments that could confirm or deny theories about the evolution of cooperation, scientists use mathematical modeling and simulation to test their hypotheses.  The game theoretic constructs known as the prisoner’s dilemma game and the donor-recipient game are often used as the basis for these investigations.
  \paragraph{}In \cite{Maloney2015a}, the eight social norms used to promote cooperation in the donor-recipient game \cite{Ohtsuki2006} were applied to the public goods game.  Those results showed that none of the social norms is evolutionary stable.  In this study, experiments are conducted to analyze the impact that population structure has on the evolution of cooperation in public goods games.
  \paragraph{}The rest of this paper is structured as follows:  section 2 covers related work on the impact of network topology on the evolution of cooperation in the prisoner’s dilemma, section 3 presents an extension of the models and methodologies presented in section 2 to the public goods game, section 4 presents the experimental results, section 5 concludes the paper and presents ideas for future work.
  
    \section{Population Structure in the Prisoner’s Dilemma Game}
    \paragraph{}This section provides an overview of related work that analyzes the impact of population structure on the evolution of cooperation in the repeated 2-person prisoner’s dilemma game.  First, a brief description of the repeated 2-person prisoner’s dilemma game is provided.  Next, studies that consider the case of a fixed network topology are reviewed.  Finally, studies that consider the case in which agents can modify the network topology are reviewed.
    
    \subsection{Prisoner’s Dilemma Game}
    \paragraph{}In the two-player prisoner’s dilemma game, each player can take one of two actions: cooperate or defect.  After each player chooses an action, the players receive payouts based on the following table:
    
    \begin{table}[h!]
      \begin{center}
      \begin{tabular}{cccc}
    	\toprule
    		&	&	\multicolumn{2}{c}{Player 2} \\
    		&   & 	Cooperate & Defect  \\ \midrule
    	\multirow{2}{*}{Player 1}
    		& Cooperate   & R  & S  \\
    		& Defect  	  & T  & P \\ \bottomrule
      \end{tabular}
      \caption{The Prisoners Dilemma Game}
	  \end{center}
    \end{table}

    \paragraph{}The payouts are constrained as follows in order to create a social dilemma:
    
    \begin{equation}
    	T > R > P > S
    \end{equation}
    
    \paragraph{}Given the order of the payouts, if the players only play the game once, regardless of the action chosen by the other player, the highest payout is always achieved by defecting.  Therefore, two rational players engaged in the game will both choose to defect and therefore receive the second lowest possible payout.  However, if both players choose to cooperate, they both receive a higher payout.  Thus, a social dilemma exists.
    \paragraph{}Rather than focusing on determining the optimal strategy to be played when two players meet in a single round of the game, evolutionary game theory considers which strategies will be most successful when the game is played during multiple interactions between agents following different strategies.  The fitness of a strategy is evaluated in terms of the payouts earned by agents that follow that strategy and is used to determine how many offspring in the next generation will follow that strategy.  In this case, evolutionary game theory tries to determine which strategies will be most evolutionarily successfully rather than which strategy is the most rational.

    \subsection{Impact of Network Topology on Evolution of Cooperation}
    \paragraph{}In this section, studies are reviewed that demonstrate the sensitivity of cooperation to the topology of the network occupied by the agents.  In the cases considered here, the network topology remains fixed during the simulation.  Although some of the studies reviewed consider the impact on cooperation in games other than the prisoner’s dilemma, this review focuses on the results reported for the prisoner’s dilemma.
    \paragraph{}All studies, except \cite{Santos2006c}, use the following payout structure that was originally used in \cite{Nowak1992}:

    \begin{align}
    	R&=1\\
    	T-b&>R\\
    	P&=S=0
    \end{align}

    \paragraph{}Given this structure for the payouts, b is the sole parameter and represents the temptation to defect.  In \cite{Santos2006c}, the authors broaden the range of payouts considered for the prisoner’s dilemma game to include values for S between zero and -1.  However, in what follows only results presented in \cite{Santos2006c} for payouts that are compatible with the structure presented above as considered.
    \paragraph{}In each study, agents are allocated to the nodes of a graph with a fixed number of nodes equal to the number of agents n.  The graph has a fixed number of edges giving the graph a fixed average connectivity z.  The agents follow one of two strategies: unconditional cooperation or unconditional defection.
    \paragraph{}For each generation, each agent plays the social dilemma game being investigated with each of its neighbors achieving a fitness score equal to the sum of the payouts earned from each game.  The number of games played by agent i is equal to the degree ki of the node it occupies and the total number of games played by all agents is equal to the number of edges in the graph.  For non-homogeneous network structures, some agents will play more games than other agents.
    \paragraph{}The evolutionary dynamics are the same as those used in \cite{Hauert2004} for simulations involving pure strategies except extended to handle graphs with heterogeneous degree.  After all games for a generation have been played, the nodes are updated synchronously.  For each agent x with payout Px, one of its neighbors y with payout Py is selected at random.  If  then agent x maintains its original strategy.  Otherwise, agent x’s strategy is switched to the strategy of agent y with probability p defined as follows:
    \begin{equation}
    	p=\frac{P_y-P_x}{\alpha\left(\max_{i\in{x,y}}k_i\right)}
    \end{equation}

    where $\alpha=max(T,R)-min(S,P)$ and $k_i$ is the degree of node $i$.

    \paragraph{}Initially, an equal number of cooperators and defectors are randomly allocated to the nodes of the graph.  After executing 10,000 generations to reach a stationary regime, the final 1000 generations are used to compute the equilibrium frequency of cooperators and defectors in the population \cite{Hauert2004}.

    \subsubsection{Fully Connected Networks}
    \paragraph{}The case of agents occupying a fully connected graph corresponds to the well-mixed population case and provides the foundation for the study of the impact of network topology on the evolution of cooperation.  A fully connected graph has an average path length of 1, a clustering coefficient of 1, an average degree $z$ of $(n-1)$ and a degree distribution consisting of a single spike located at $z$.
    \paragraph{} In \cite{Santos2006c}, the authors run experiments on well-mixed populations of agents that occupy a complete graph.  The authors reconfirm (need references) that cooperators are driven to extinction when playing the prisoner’s dilemma in a well-mixed population.
    
    \subsubsection{Homogeneous Graphs}
    \paragraph{}A fully connected graph is a special case of a regular graph in which every node is connected to every other node giving it an average degree equal to $(n-1)$.   Homogeneous regular graphs with lower average degree have greater average path lengths and lower clustering coefficients than fully connected graphs while preserving a spiked degree distribution.  The regular structure of the graph creates correlations between vertices such that the neighbors of a node are more likely to be neighbors of each other.  This causes the clustering coefficient to be higher than for an unstructured graph with the same average degree.  The increased average path length combined with the moderate clustering coefficient provides the opportunity for some separation between clusters of nodes compared to the fully connected case in which the graph contains a single cluster of nodes.
    \paragraph{}The authors of \cite{Pacheco2005}, \cite{Santos2006a}, \cite{Santos2006c} and \cite{Santos2005b} find that when the agents occupy a regular graph with average degree significantly less than $(n-1)$, cooperators can eliminate defectors for small values of b.  This small window of opportunity exists because the regular spatial structure allows cooperators to form compact clusters that resist invasion by defectors.  As expected, increasing $b$ decreases the performance of cooperators.  In addition, as the average degree of the graph increases the population structure begins to mirror a well-mixed population and cooperation becomes more difficult.
    \paragraph{}A homogeneous regular graph can be transformed into a homogeneous random graph by swapping the ends of randomly selected edges until all edges have had their ends swapped while ensuring that no duplicate edges or loops have been introduced \cite{Maslov2002}.  This procedure leaves the degree of each node and the degree distribution of the graph unchanged. However, randomly swapping the ends of edges removes vertex correlations and introduces “short cuts” which reduces both the clustering coefficient and the average path length of the resulting graph compared to the original graph.  This reduces the possibility of the formation of small semi-isolated clusters of nodes.
    \paragraph{}In \cite{Santos2006c}, the authors find that when occupying the nodes of a homogeneous random graph, the best cooperators can do is coexist with defectors for small values of $b$.  The decrease in cooperation is because the uncorrelated social structure no longer allows cooperators to form tight clusters that resist invasion by defectors.  However, the cooperation level is higher than on fully connected graphs demonstrating that the reduction in degree provides some ability for cooperators to resist defectors at least temporarily.
    \paragraph{} The results presented in this section show that a regular graph structure introduces correlations between nodes in the graph that allow agents to form tight clusters that resist invasion by defectors at least for small values of $b$.  However, regular graphs are unrealistic representations of most real world networks \cite{Pacheco2005}.  The following section reviews some graph topologies that provide better representations of real networks and also provide improved conditions for the evolution of cooperation.

    \subsubsection{Heterogeneous Graphs}
	\paragraph{}In the previous sections, vertex correlations were shown to benefit cooperators by enabling the formation of compact clusters.  In this section, the impact of heterogeneous degree distribution on the evolution of cooperation is investigated.
	\paragraph{}A small-world graph is a graph that has moderate clustering coefficients similar to regular graphs and small average path lengths similar to random graphs \cite{Watts1998}.  The degree distribution of these small-world networks is moderately heterogeneous.  The Watts-Strogatz algorithm \cite{Watts1998} can be used to construct graphs that range from regular graphs to random graphs with the graphs in the middle of this spectrum displaying small-world characteristics.  Starting from a regular ring graph, the final graph is produced by randomly rewiring each edge in the graph with probability p.
	\paragraph{}The parameter p can be used to control the degree of heterogeneity introduced into the original graph.  For p=0, the original random graph is preserved.  For p<0.01, the average path length of the graph decreases quickly as p increases while the clustering coefficient remains virtually unchanged.  For p>0.01, the clustering coefficient decreases rapidly while the average path length remains virtually unchanged.  As p approaches 1, the graph structure approaches a random graph.
	\paragraph{}In the studies cited in the previous section, the authors found that cooperators that occupy regular graphs can coexist with defectors for values of b up to1.1 when the average degree of the graph is kept low.  For small-world networks, in \cite{Pacheco2005} and \cite{Santos2006b}, the authors find that cooperators can coexist with defectors over a broader range of values for b.  For p=0.1, cooperators can coexist for values of b up to approximately 1.2.  As p increases, cooperators can coexist with defectors for larger values of b until a maximum of approximately 1.6 is reached for p=1 \cite{Santos2006b}.  As with the case of a regular graph, cooperation suffers as the average connectivity z of the graph increases and the population approaches a well-mixed population.
	\paragraph{}These results indicate that cooperators fare best on small-world networks when the clustering coefficient and average path length are at their smallest.  This seems to contradict the results reported in the previous section where decreasing clustering coefficient and decreasing average path length reduced the ability of cooperators to form compact clusters and thus reduced their ability to compete with defectors.  It appears that the success of cooperators is due to the heterogeneous degree distributions possessed by these networks.
	\paragraph{}In \cite{Santos2005a}, the authors introduce a homogeneous small-world graph in order to better understand the impact of heterogeneity on the evolution of cooperation.  To construct a homogeneous small-world graph, the authors modify the Watts-Strogatz algorithm so that the degree of each node is preserved during the rewiring process.  A parameter f, representing the fraction of edges to be rewired, plays a role similar to the role played by p in the original algorithm.  The authors report that the average path length and clustering coefficient vary relative to f in the same way they vary relative to p indicating that the resulting graphs provide an reasonable representation of a homogeneous small-world graph.
	\paragraph{}The authors find that the performance of cooperators on homogeneous small-world graphs is worse than their performance on regular graphs for small values of b and only marginally better for larger values of b.  Performance of cooperators on heterogeneous small-world graphs is significantly better than on the homogeneous variety for all values of b.  This indicates that most of the performance improvement of cooperators on heterogeneous small-world graphs reported earlier in this section is due to the heterogeneity of these graphs rather than their small-world properties.
	\paragraph{}To further investigate the impacts of heterogeneity on the evolution of cooperation, the authors in \cite{Santos2006c} investigate the performance of cooperators on moderately heterogeneous random graphs called single scale-graphs \cite{Amaral2000}.  These graphs are generated using the configuration model \cite{Molloy1995} to produce a maximally random graph that is compatible with the specified degree distribution.  The degree distribution of these graphs is more heterogeneous than the small-world graphs considered earlier in this section.
	\paragraph{}The authors find that cooperators perform significantly better on single scale networks than on regular graphs.  Comparison of the results reported in \cite{Santos2006c} with those reported in \cite{Pacheco2005} and \cite{Santos2005a} show that cooperators fare about the same on single scale networks as on Watts-Strogatz graphs with p=1.  For single scale graphs, cooperators can eliminate defectors for values of b up to approximately 1.16 and coexist up to 1.5.  This once again points to heterogeneity providing a significant boost to cooperation.
	\paragraph{}The results presented in this section indicate that heterogeneity can overcome the problems introduced by low average path length and low clustering coefficients.  In a heterogeneous network, some nodes in the network allow agents to participate in more interactions than others.  These hub nodes give their occupants the ability to accumulate higher payouts and therefore give them a fitness advantage
	\paragraph{}Both cooperators and defectors can benefit from occupying a hub but, in this case, defectors become victims of their own success.  The strategy of the agent that occupies the hub will be propagated to the agents in the neighboring nodes.  This means that defectors will surround a defector while cooperators will surround a cooperator.  However, both defectors and cooperators perform best when surrounded by cooperators.  As more and more defectors surround an incumbent defector, its fitness decreases until a cooperator is able to take over the hub.  Once a cooperator holds the hub, its strategy is propagated to neighboring nodes reinforcing its dominant position and driving defectors to nodes with moderate to low connectivity \cite{Pacheco2005}\cite{Santos2006a}\cite{Santos2006b}.
	\paragraph{}The following section reviews graph topologies that improve upon the benefits to cooperation provided by the topologies considered in this section by increasing heterogeneity and reintroducing vertex correlations.

    \section{Temporary Citations}
    \paragraph{}Here are some citations: \cite{Axelrod1981}\cite{Nowak1992}\cite{Hauert2004}\cite{Pacheco2005}\cite{Santos2006b}\cite{Santos2006a}\cite{Santos2005a}\cite{Santos2006c}\cite{Santos2005b}\cite{Eguiluz2005}\cite{Santos2006d}\cite{Fu2008}\cite{Traulsen2006}\cite{Watts1998}\cite{Barabasi1999}\cite{Amaral2000}\cite{Molloy1995}\cite{Maslov2002}\cite{Dorogovtsev2003}\cite{Dorogovtsev2001}\cite{Nowak1998}\cite{Boyd1988}\cite{Boyd1992}\cite{Hauert2002}\cite{Hauert2007}\cite{Li2014}\cite{Maloney2015a}\cite{Ohtsuki2006}\cite{Macy1991}.
    
%    \section{References}
    \bibliography{../references}
    \bibliographystyle{ieeetr}
\end{document}